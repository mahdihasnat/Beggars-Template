\documentclass{article}
\usepackage[utf8x]{inputenc}
\usepackage[english]{babel}
\usepackage[landscape,a4paper]{geometry}
\geometry{top=0.7in,bottom=0.3in,left=0.2in,right=0.2in}
\usepackage{amsmath}

% \usepackage{color}   %May be necessary if you want to color links
\usepackage{hyperref}
\hypersetup{
    colorlinks=false, %set true if you want colored links
    linktoc=all,     %set to all if you want both sections and subsections linked
    linkcolor=blue,  %choose some color if you want links to stand out
}

\usepackage{multicol}
\setlength{\columnseprule}{1pt}
\def\columnseprulecolor{\color{black}}

\usepackage[cache=true]{minted}
\usemintedstyle{vs}
\setminted{frame=bottomline,framesep=1mm,fontsize=\normalsize,tabsize=2,breaklines,baselinestretch=0.89}

\usepackage{lastpage}
\usepackage{fancyhdr}
\pagestyle{fancy}
\fancyhf{}
\rhead{Page \thepage/\pageref{LastPage}}
\lhead{BUET XYZ -- Bangladesh University of Engineering and Technology}


\begin{document}
\begin{multicols}{3}[]
\begin{tableofcontents}
\end{tableofcontents}

\section{DP}

\subsection{DSU on Tree}

\begin{minted}{cpp}
#include<bits/stdc++.h>
#include<math.h>
#include<vector>
using namespace std;
#define MAX 100005
#define MOD 1000000007
vector<int> *pvec[MAX];
vector<int> Graph[MAX];
int subtree[MAX];
int color[MAX];
int color_counter[MAX];
pair<long long int,int> Info[MAX];
int Subtree(int node,int parent=-1)
{
    subtree[node]=1;
    int i;
    for(i=0; i<Graph[node].size(); i++)
    {
        if(Graph[node][i]==parent) continue;
        subtree[node]=subtree[node]+Subtree(Graph[node][i],node);
    }
    return subtree[node];
}
pair<long long int,int> dfs(int node,int parent=-1,bool keep=false)
{
    int i,j,k,child,hchild=-1;
    for(i=0; i<Graph[node].size(); i++)
    {
        if(Graph[node][i]==parent) continue;
        if(hchild==-1||subtree[hchild]<subtree[Graph[node][i]])
        {
            hchild=Graph[node][i];
        }
    }
    for(i=0; i<Graph[node].size(); i++)
    {
        if(Graph[node][i]==parent||Graph[node][i]==hchild) continue;
        dfs(Graph[node][i],node,false);
    }
    if(hchild!=-1)
    {
        Info[node]=dfs(hchild,node,true);
        pvec[node]=pvec[hchild];
    }
    else
    {
        pvec[node]=new vector<int> ();
    }
    pvec[node]->push_back(node);
    color_counter[color[node]]++;
    if(color_counter[color[node]]>Info[node].second)
    {
        Info[node].second=color_counter[color[node]];
        Info[node].first=color[node];
    }
    else if(color_counter[color[node]]==Info[node].second)
    {
        Info[node].first=Info[node].first+color[node];
    }
    for(i=0; i<Graph[node].size(); i++)
    {
        if(Graph[node][i]==parent||Graph[node][i]==hchild) continue;
        child=Graph[node][i];
        for(j=0; j<(*pvec[child]).size(); j++)
        {
            k=(*pvec[child])[j];
            pvec[node]->push_back(k);
            color_counter[color[k]]++;
            if(color_counter[color[k]]>Info[node].second)
            {
                Info[node].second=color_counter[color[k]];
                Info[node].first=color[k];
            }
            else if(color_counter[color[k]]==Info[node].second)
            {
                Info[node].first=Info[node].first+color[k];
            }
        }
    }
    if(!keep)
    {
        for(j=0; j<(*pvec[node]).size(); j++)
        {
            k=(*pvec[node])[j];
            color_counter[color[k]]--;
        }
    }
    return Info[node];
}
int main()
{
    int n,u,v,i;
    scanf("%d",&n);
    for(i=1; i<=n; i++)
    {
        scanf("%d",&color[i]);
    }
    for(i=1; i<n; i++)
    {
        scanf("%d %d",&u,&v);
        Graph[u].push_back(v);
        Graph[v].push_back(u);
    }
    Subtree(1);
    dfs(1);
    for(i=1; i<=n; i++)
    {
        printf("%I64d ",Info[i].first);
    }
    return 0;
}

\end{minted}


\subsection{Divide and Conquer Optimization}

\begin{minted}{cpp}
int m, n;
vector<long long> dp_before(n), dp_cur(n);

long long C(int i, int j);

// compute dp_cur[l], ... dp_cur[r] (inclusive)
void compute(int l, int r, int optl, int optr) {
    if (l > r)
        return;

    int mid = (l + r) >> 1;
    pair<long long, int> best = {LLONG_MAX, -1};

    for (int k = optl; k <= min(mid, optr); k++) {
        best = min(best, {(k ? dp_before[k - 1] : 0) + C(k, mid), k});
    }

    dp_cur[mid] = best.first;
    int opt = best.second;

    compute(l, mid - 1, optl, opt);
    compute(mid + 1, r, opt, optr);
}

int solve() {
    for (int i = 0; i < n; i++)
        dp_before[i] = C(0, i);

    for (int i = 1; i < m; i++) {
        compute(0, n - 1, 0, n - 1);
        dp_before = dp_cur;
    }

    return dp_before[n - 1];
}

\end{minted}


\subsection{Li Chao Tree}

\begin{minted}{cpp}
#include <bits/stdc++.h>
#include <vector>
#include<math.h>
#include<string.h>
using namespace std;
#define MAX 200005
#define MOD 1000000007
#define INF 10000000000
#define EPS 0.0000000001
#define FASTIO ios_base::sync_with_stdio(false);cin.tie(NULL)
#include <ext/pb_ds/assoc_container.hpp>
#include <ext/pb_ds/tree_policy.hpp>
#include <ext/pb_ds/detail/standard_policies.hpp>
class LiChaoTree
{
    long long L,R;
    bool minimize;
    int lines;
    struct Node
    {
        complex<long long> line;
        Node *children[2];
        Node(complex<long long> ln= {0,1000000000000000000})
        {
            line=ln;
            children[0]=0;
            children[1]=0;
        }
    } *root;
    long long dot(complex<long long> a, complex<long long> b)
    {
        return (conj(a) * b).real();
    }
    long long f(complex<long long> a,  long long x)
    {
        return dot(a, {x, 1});
    }
    void clear(Node* &node)
    {
        if(node->children[0])
        {
            clear(node->children[0]);
        }
        if(node->children[1])
        {
            clear(node->children[1]);
        }
        delete node;
    }
    void add_line(complex<long long> nw, Node* &node, long long l, long long r)
    {
        if(node==0)
        {
            node=new Node(nw);
            return;
        }
        long long m = (l + r) / 2;
        bool lef = (f(nw, l) < f(node->line, l)&&minimize)||((!minimize)&&f(nw, l) > f(node->line, l));
        bool mid = (f(nw, m) < f(node->line, m)&&minimize)||((!minimize)&&f(nw, m) > f(node->line, m));
        if(mid)
        {
            swap(node->line, nw);
        }
        if(r - l == 1)
        {
            return;
        }
        else if(lef != mid)
        {
            add_line(nw, node->children[0], l, m);
        }
        else
        {
            add_line(nw, node->children[1], m, r);
        }
    }
    long long get(long long x, Node* &node, long long l, long long r)
    {
        long long m = (l + r) / 2;
        if(r - l == 1)
        {
            return f(node->line, x);
        }
        else if(x < m)
        {
            if(node->children[0]==0) return f(node->line, x);
            if(minimize) return min(f(node->line, x), get(x, node->children[0], l, m));
            else return max(f(node->line, x), get(x, node->children[0], l, m));
        }
        else
        {
            if(node->children[1]==0) return f(node->line, x);
            if(minimize) return min(f(node->line, x), get(x, node->children[1], m, r));
            else return max(f(node->line, x), get(x, node->children[1], m, r));
        }
    }
public:
    LiChaoTree(long long l=-1000000001,long long r=1000000001,bool mn=false)
    {
        L=l;
        R=r;
        root=0;
        minimize=mn;
        lines=0;
    }
    void AddLine(pair<long long,long long> ln)
    {
        add_line({ln.first,ln.second},root,L,R);
        lines++;
    }
    int number_of_lines()
    {
        return lines;
    }
    long long getOptimumValue(long long x)
    {
        return get(x,root,L,R);
    }
    ~LiChaoTree()
    {
        if(root!=0) clear(root);
    }
};
int main()
{
    return 0;
}

\end{minted}


\section{DS}

\subsection{CD-anikda}

\begin{minted}{cpp}
// p[u] = parent of u in centroid tree
// d[x][u] = distance from u to a parent of u at level x of centroid tree
//           if u is in subtree of centroid c, then d[lvl[c]][u] = dist(c, l)
// Taken from Rezwan Arefin
// If (x, y) edge exist, then x must be in adj[y] and y must be in adj[x]
const int maxn = 1e5 + 10; 
vector<int> adj[maxn]; 
int lvl[maxn], sub[maxn], p[maxn], vis[maxn], d[18][maxn], ans[maxn];
	
void calc(int u, int par) { sub[u] = 1; 
	for(int v : adj[u]) if(v - par && !vis[v]) 
		calc(v, u), sub[u] += sub[v];
}
int centroid(int u, int par, int r) {
	for(int v : adj[u]) if(v - par && !vis[v]) 
		if(sub[v] > r) return centroid(v, u, r);
	return u;
}
void dfs(int l, int u, int par) {
	if(par + 1) d[l][u] = d[l][par] + 1; 
	for(int v : adj[u]) if(v - par && !vis[v]) 
		dfs(l, v, u);
}
void decompose(int u, int par) {
	calc(u, -1);
	int c = centroid(u, -1, sub[u] >> 1);
	vis[c] = 1, p[c] = par, lvl[c] = 0; 
	if(par + 1) lvl[c] = lvl[par] + 1;
	dfs(lvl[c], c, -1);
	for(int v : adj[c]) if(v - par && !vis[v]) 
		decompose(v, c);
}

void update(int u) {
	for(int v = u; v + 1; v = p[v]) 
		ans[v] = min(ans[v], d[lvl[v]][u]);
}
int query(int u) {
	int ret = 1e9;
	for(int v = u; v + 1; v = p[v])
		ret = min(ret, ans[v] + d[lvl[v]][u]);
	return ret;
}

\end{minted}


\subsection{HLD-anikda}

\begin{minted}{cpp}
LazySegmentTree Tree ; 
int sz[MAX];
int in[MAX];
int rin[MAX];
int out[MAX];
int head[MAX];
int par[MAX];
vector<int>g[MAX];
void dfs_sz(int u,int p) {
    sz[u] = 1;
    par[u] = p;
    for(auto &v: g[u]) {
        if(v==p)continue;
        dfs_sz(v,u);
        sz[u] += sz[v];
        if(sz[v] > sz[g[u][0]])
            swap(v,g[u][0]);
    }
}
int t;
void dfs_hld(int u,int p) {
    in[u] = ++t;
    rin[in[u]] = u;
    for(auto v: g[u]) {
        if(v==p)continue;
        head[v] = (v == g[u][0] ? head[u] : v);
        dfs_hld(v,u);
    }
    out[u] = t;
}
bool isParent(int p,int u){
    return in[p]<=in[u]&&out[u]<=out[p];
}
int n ;
int pathQuery(int u,int v){
    int ret = -inf;
    while(true){
        if(isParent(head[u],v))break;
        ret=max(ret,Tree.queryRange(1,1,n,in[head[u]],in[u]));
        u=par[head[u]];
    }
    swap(u,v);
    while(true){
        if(isParent(head[u],v))break;
        ret=max(ret,Tree.queryRange(1,1,n,in[head[u]],in[u]));
        u=par[head[u]];
    }
    if(in[v]<in[u])swap(u,v);
    ret = max(ret,Tree.queryRange(1,1,n,in[u],in[v]));
    return ret;
}
void updateSubTree(int u,int val){
    Tree.updateRange(1,1,n,in[u],out[u],val);
}
void buildHLD(int root){
    dfs_sz(root,root);
    head[root]=root;
    dfs_hld(root,root);
}
// call buildHLD
\end{minted}


\subsection{Implicit Treap}

\begin{minted}{cpp}
#include<bits/stdc++.h>
#include<math.h>
#include<vector>
#include<stdlib.h>
using namespace std;
#define MAX 200005
#define MOD 998244353
#define NINF -1000000000000000000
template <class T>
class implicit_treap
{
    struct item
    {
        int prior, cnt;
        T value;
        bool rev;
        item *l,*r;
        item(T v)
        {
            value=v;
            rev=false;
            l=NULL;
            r=NULL;
            cnt=1;
            prior=rand();
        }
    } *root,*node;
    int cnt (item * it)
    {
        return it ? it->cnt : 0;
    }

    void upd_cnt (item * it)
    {
        if (it)
            it->cnt = cnt(it->l) + cnt(it->r) + 1;
    }

    void push (item * it)
    {
        if (it && it->rev)
        {
            it->rev = false;
            swap (it->l, it->r);
            if (it->l)  it->l->rev ^= true;
            if (it->r)  it->r->rev ^= true;
        }
    }

    void merge (item * & t, item * l, item * r)
    {
        push (l);
        push (r);
        if (!l || !r)
            t = l ? l : r;
        else if (l->prior > r->prior)
            merge (l->r, l->r, r),  t = l;
        else
            merge (r->l, l, r->l),  t = r;
        upd_cnt (t);
    }

    void split (item * t, item * & l, item * & r, int key, int add = 0)
    {
        if (!t)
            return void( l = r = 0 );
        push (t);
        int cur_key = add + cnt(t->l);
        if (key <= cur_key)
            split (t->l, l, t->l, key, add),  r = t;
        else
            split (t->r, t->r, r, key, add + 1 + cnt(t->l)),  l = t;
        upd_cnt (t);
    }
    void insert(item * &t,item * element,int key)
    {
        item *l,*r;
        split(t,l,r,key);
        merge(l,l,element);
        merge(t,l,r);
        l=NULL;
        r=NULL;
    }
    T elementAt(item * &t,int key)
    {
        push(t);
        T ans;
        if(cnt(t->l)==key) ans=t->value;
        else if(cnt(t->l)>key) ans=elementAt(t->l,key);
        else ans=elementAt(t->r,key-1-cnt(t->l));
        return ans;
    }
    void erase (item * & t, int key)
    {
        push(t);
        if(!t) return;
        if (key == cnt(t->l))
            merge (t, t->l, t->r);
        else if(key<cnt(t->l))
            erase(t->l,key);
        else
            erase(t->r,key-cnt(t->l)-1);
        upd_cnt(t);
    }
    void reverse (item * &t, int l, int r)
    {
        item *t1, *t2, *t3;
        split (t, t1, t2, l);
        split (t2, t2, t3, r-l+1);
        t2->rev ^= true;
        merge (t, t1, t2);
        merge (t, t, t3);
    }
    void cyclic_shift(item * &t,int L,int R)
    {
        if(L==R) return;
        item *l,*r,*m;
        split(t,t,l,L);
        split(l,l,m,R-L+1);
        split(l,l,r,R-L);
        merge(t,t,r);
        merge(t,t,l);
        merge(t,t,m);
        l=NULL;
        r=NULL;
        m=NULL;
    }
    void output (item * t,vector<T> &arr)
    {
        if (!t)  return;
        push (t);
        output (t->l,arr);
        arr.push_back(t->value);
        output (t->r,arr);
    }
public:
    implicit_treap()
    {
        root=NULL;
    }
    void insert(T value,int position)
    {
        node=new item(value);
        insert(root,node,position);
    }
    void erase(int position)
    {
        erase(root,position);
    }
    void reverse(int l,int r)
    {
        reverse(root,l,r);
    }
    T elementAt(int position)
    {
        return elementAt(root,position);
    }
    void cyclic_shift(int L,int R)
    {
        cyclic_shift(root,L,R);
    }
    int size()
    {
        return cnt(root);
    }
    void output(vector<T> &arr)
    {
        output(root,arr);
    }
};

\end{minted}


\subsection{Mo Algorithm}

\begin{minted}{cpp}
#include<bits/stdc++.h>
using namespace std;
#define MOD 998244353
#define MAX 200005
#define MAX_BIT 50
#define PRECISION 0.00000000001
#define INF 2000000000
void remove(int idx);  // TODO: remove value at idx from data structure
void add(int idx);     // TODO: add value at idx from data structure
int get_answer();  // TODO: extract the current answer of the data structure
int block_size;
struct Query {
    int l, r,k, idx;
    bool operator<(Query other) const
    {
        if(l/block_size!=other.l/block_size) return (l<other.l);
        return (l/block_size&1)? (r<other.r) : (r>other.r);
    }
};
vector<int> mo_s_algorithm(vector<Query> queries) {
    vector<int> answers(queries.size());
    sort(queries.begin(), queries.end());

    // TODO: initialize data structure

    int cur_l = 0;
    int cur_r = -1;
    // invariant: data structure will always reflect the range [cur_l, cur_r]
    for (Query q : queries) {
        while (cur_l > q.l) {
            cur_l--;
            add(cur_l);
        }
        while (cur_r < q.r) {
            cur_r++;
            add(cur_r);
        }
        while (cur_l < q.l) {
            remove(cur_l);
            cur_l++;
        }
        while (cur_r > q.r) {
            remove(cur_r);
            cur_r--;
        }
        answers[q.idx] = get_answer();
    }
    return answers;
}

int main()
{
    return 0;
}

\end{minted}


\subsection{Treap}

\begin{minted}{cpp}
#include<bits/stdc++.h>
#include<math.h>
#include<vector>
#include<stdlib.h>
using namespace std;
#define MAX 400005
#define MOD 998244353
#define INF 2000000000
template <class T>
class treap
{
    struct item
    {
        int prior, cnt;
        T key;
        item *l,*r;
        item(T v)
        {
            key=v;
            l=NULL;
            r=NULL;
            cnt=1;
            prior=rand();
        }
    } *root,*node;
    int cnt (item * it)
    {
        return it ? it->cnt : 0;
    }

    void upd_cnt (item * it)
    {
        if (it)
            it->cnt = cnt(it->l) + cnt(it->r) + 1;
    }

    void split (item * t, T key, item * & l, item * & r)
    {
        if (!t)
            l = r = NULL;
        else if (key < t->key)
            split (t->l, key, l, t->l),  r = t;
        else
            split (t->r, key, t->r, r),  l = t;
        upd_cnt(t);
    }

    void insert (item * & t, item * it)
    {
        if (!t)
            t = it;
        else if (it->prior > t->prior)
            split (t, it->key, it->l, it->r),  t = it;
        else
            insert (it->key < t->key ? t->l : t->r, it);
        upd_cnt(t);
    }

    void merge (item * & t, item * l, item * r)
    {
        if (!l || !r)
            t = l ? l : r;
        else if (l->prior > r->prior)
            merge (l->r, l->r, r),  t = l;
        else
            merge (r->l, l, r->l),  t = r;
        upd_cnt(t);
    }

    void erase (item * & t, T key)
    {
        if (t->key == key)
            merge (t, t->l, t->r);
        else
            erase (key < t->key ? t->l : t->r, key);
        upd_cnt(t);
    }

    T elementAt(item * &t,int key)
    {
        T ans;
        if(cnt(t->l)==key) ans=t->key;
        else if(cnt(t->l)>key) ans=elementAt(t->l,key);
        else ans=elementAt(t->r,key-1-cnt(t->l));
        upd_cnt(t);
        return ans;
    }

    item * unite (item * l, item * r)
    {
        if (!l || !r)  return l ? l : r;
        if (l->prior < r->prior)  swap (l, r);
        item * lt, * rt;
        split (r, l->key, lt, rt);
        l->l = unite (l->l, lt);
        l->r = unite (l->r, rt);
        upd_cnt(l);
        upd_cnt(r);
        return l;
    }
    void heapify (item * t)
    {
        if (!t) return;
        item * max = t;
        if (t->l != NULL && t->l->prior > max->prior)
            max = t->l;
        if (t->r != NULL && t->r->prior > max->prior)
            max = t->r;
        if (max != t)
        {
            swap (t->prior, max->prior);
            heapify (max);
        }
    }

    item * build (T * a, int n)
    {
        if (n == 0) return NULL;
        int mid = n / 2;
        item * t = new item (a[mid], rand ());
        t->l = build (a, mid);
        t->r = build (a + mid + 1, n - mid - 1);
        heapify (t);
        return t;
    }
    void output (item * t,vector<T> &arr)
    {
        if (!t)  return;
        output (t->l,arr);
        arr.push_back(t->key);
        output (t->r,arr);
    }
public:
    treap()
    {
        root=NULL;
    }
    treap(T *a,int n)
    {
        build(a,n);
    }
    void insert(T value)
    {
        node=new item(value);
        insert(root,node);
    }
    void erase(T value)
    {
        erase(root,value);
    }
    T elementAt(int position)
    {
        return elementAt(root,position);
    }
    int size()
    {
        return cnt(root);
    }
    void output(vector<T> &arr)
    {
        output(root,arr);
    }
    int range_query(T l,T r) //(l,r]
    {
        item *previous,*next,*current;
        split(root,l,previous,current);
        split(current,r,current,next);
        int ans=cnt(current);
        merge(root,previous,current);
        merge(root,root,next);
        previous=NULL;
        current=NULL;
        next=NULL;
        return ans;
    }
};

\end{minted}


\section{Flow}

\subsection{Dinic's Algorithm}

\begin{minted}{cpp}
#include<bits/stdc++.h>
#include<vector>
using namespace std;
#define MAX 100
#define HUGE_FLOW 1000000000
#define BEGIN 1
#define DEFAULT_LEVEL 0
struct FlowEdge {
    int v, u;
    long long cap, flow = 0;
    FlowEdge(int v, int u, long long cap) : v(v), u(u), cap(cap) {}
};

struct Dinic {
    const long long flow_inf = 1e18;
    vector<FlowEdge> edges;
    vector<vector<int>> adj;
    int n, m = 0;
    int s, t;
    vector<int> level, ptr;
    queue<int> q;

    Dinic(int n, int s, int t) : n(n), s(s), t(t) {
        adj.resize(n);
        level.resize(n);
        ptr.resize(n);
    }

    void add_edge(int v, int u, long long cap) {
        edges.emplace_back(v, u, cap);
        edges.emplace_back(u, v, 0);
        adj[v].push_back(m);
        adj[u].push_back(m + 1);
        m += 2;
    }

    bool bfs() {
        while (!q.empty()) {
            int v = q.front();
            q.pop();
            for (int id : adj[v]) {
                if (edges[id].cap - edges[id].flow < 1)
                    continue;
                if (level[edges[id].u] != -1)
                    continue;
                level[edges[id].u] = level[v] + 1;
                q.push(edges[id].u);
            }
        }
        return level[t] != -1;
    }

    long long dfs(int v, long long pushed) {
        if (pushed == 0)
            return 0;
        if (v == t)
            return pushed;
        for (int& cid = ptr[v]; cid < (int)adj[v].size(); cid++) {
            int id = adj[v][cid];
            int u = edges[id].u;
            if (level[v] + 1 != level[u] || edges[id].cap - edges[id].flow < 1)
                continue;
            long long tr = dfs(u, min(pushed, edges[id].cap - edges[id].flow));
            if (tr == 0)
                continue;
            edges[id].flow += tr;
            edges[id ^ 1].flow -= tr;
            return tr;
        }
        return 0;
    }

    long long flow() {
        long long f = 0;
        while (true) {
            fill(level.begin(), level.end(), -1);
            level[s] = 0;
            q.push(s);
            if (!bfs())
                break;
            fill(ptr.begin(), ptr.end(), 0);
            while (long long pushed = dfs(s, flow_inf)) {
                f += pushed;
            }
        }
        return f;
    }
};
int main()
{
    return 0;
}

\end{minted}


\subsection{Edmond's Blossom Algorithm}

\begin{minted}{cpp}
/***Copied from https://codeforces.com/blog/entry/49402***/

/*
GETS:
V->number of vertices
E->number of edges
pair of vertices as edges (vertices are 1..V)

GIVES:
output of edmonds() is the maximum matching
match[i] is matched pair of i (-1 if there isn't a matched pair)
 */

#include <bits/stdc++.h>
using namespace std;
const int M=500;
struct struct_edge
{
    int v;
    struct_edge* n;
};
typedef struct_edge* edge;
struct_edge pool[M*M*2];
edge top=pool,adj[M];
int V,E,match[M],qh,qt,q[M],father[M],base[M];
bool inq[M],inb[M],ed[M][M];
void add_edge(int u,int v)
{
    top->v=v,top->n=adj[u],adj[u]=top++;
    top->v=u,top->n=adj[v],adj[v]=top++;
}
int LCA(int root,int u,int v)
{
    static bool inp[M];
    memset(inp,0,sizeof(inp));
    while(1)
    {
        inp[u=base[u]]=true;
        if (u==root) break;
        u=father[match[u]];
    }
    while(1)
    {
        if (inp[v=base[v]]) return v;
        else v=father[match[v]];
    }
}
void mark_blossom(int lca,int u)
{
    while (base[u]!=lca)
    {
        int v=match[u];
        inb[base[u]]=inb[base[v]]=true;
        u=father[v];
        if (base[u]!=lca) father[u]=v;
    }
}
void blossom_contraction(int s,int u,int v)
{
    int lca=LCA(s,u,v);
    memset(inb,0,sizeof(inb));
    mark_blossom(lca,u);
    mark_blossom(lca,v);
    if (base[u]!=lca)
        father[u]=v;
    if (base[v]!=lca)
        father[v]=u;
    for (int u=0; u<V; u++)
        if (inb[base[u]])
        {
            base[u]=lca;
            if (!inq[u])
                inq[q[++qt]=u]=true;
        }
}
int find_augmenting_path(int s)
{
    memset(inq,0,sizeof(inq));
    memset(father,-1,sizeof(father));
    for (int i=0; i<V; i++) base[i]=i;
    inq[q[qh=qt=0]=s]=true;
    while (qh<=qt)
    {
        int u=q[qh++];
        for (edge e=adj[u]; e; e=e->n)
        {
            int v=e->v;
            if (base[u]!=base[v]&&match[u]!=v)
                if ((v==s)||(match[v]!=-1 && father[match[v]]!=-1))
                    blossom_contraction(s,u,v);
                else if (father[v]==-1)
                {
                    father[v]=u;
                    if (match[v]==-1)
                        return v;
                    else if (!inq[match[v]])
                        inq[q[++qt]=match[v]]=true;
                }
        }
    }
    return -1;
}
int augment_path(int s,int t)
{
    int u=t,v,w;
    while (u!=-1)
    {
        v=father[u];
        w=match[v];
        match[v]=u;
        match[u]=v;
        u=w;
    }
    return t!=-1;
}
int edmonds()
{
    int matchc=0;
    memset(match,-1,sizeof(match));
    for (int u=0; u<V; u++)
        if (match[u]==-1)
            matchc+=augment_path(u,find_augmenting_path(u));
    return matchc;
}
int main()
{
    FILE *in=stdin;
    int u,v;
    fscanf(in,"%d",&V);
    while(fscanf(in,"%d %d",&u,&v)!=EOF)
    {

        if (!ed[u-1][v-1])
        {
            add_edge(u-1,v-1);
            ed[u-1][v-1]=ed[v-1][u-1]=true;
        }
    }
    printf("%d\n",2*edmonds());
    for (int i=0; i<V; i++)
        if (i<match[i])
            printf("%d %d\n",i+1,match[i]+1);
    return 0;
}

\end{minted}


\subsection{Hungarian Algorithm}

\begin{minted}{cpp}
#include <bits/stdc++.h>
#include <vector>
#include<math.h>
#include<string.h>
using namespace std;
#define MAX 300005
#define MOD 1000000007
#define GMAX 19
#define INF 20000000000000000
#define EPS 0.000000001
#define FASTIO ios_base::sync_with_stdio(false);cin.tie(NULL)
#include <ext/pb_ds/assoc_container.hpp>
#include <ext/pb_ds/tree_policy.hpp>
#include <ext/pb_ds/detail/standard_policies.hpp>
class HungarianAlgorithm
{
    int N,inf,n,max_match;
    int *lx,*ly,*xy,*yx,*slack,*slackx,*prev;
    int **cost;
    bool *S,*T;
    void init_labels()
    {
        for(int x=0;x<n;x++) lx[x]=0;
        for(int y=0;y<n;y++) ly[y]=0;
        for (int x = 0; x < n; x++)
            for (int y = 0; y < n; y++)
                lx[x] = max(lx[x], cost[x][y]);
    }
    void update_labels()
    {
        int x, y, delta = inf; //init delta as infinity
        for (y = 0; y < n; y++) //calculate delta using slack
            if (!T[y])
                delta = min(delta, slack[y]);
        for (x = 0; x < n; x++) //update X labels
            if (S[x]) lx[x] -= delta;
        for (y = 0; y < n; y++) //update Y labels
            if (T[y]) ly[y] += delta;
        for (y = 0; y < n; y++) //update slack array
            if (!T[y])
                slack[y] -= delta;
    }
    void add_to_tree(int x, int prevx)
//x - current vertex,prevx - vertex from X before x in the alternating path,
//so we add edges (prevx, xy[x]), (xy[x], x)
    {
        S[x] = true; //add x to S
        prev[x] = prevx; //we need this when augmenting
        for (int y = 0; y < n; y++) //update slacks, because we add new vertex to S
            if (lx[x] + ly[y] - cost[x][y] < slack[y])
            {
                slack[y] = lx[x] + ly[y] - cost[x][y];
                slackx[y] = x;
            }
    }
    void augment() //main function of the algorithm
    {
        if (max_match == n) return; //check wether matching is already perfect
        int x, y, root; //just counters and root vertex
        int q[N], wr = 0, rd = 0; //q - queue for bfs, wr,rd - write and read
//pos in queue
        //memset(S, false, sizeof(S)); //init set S
        for(int i=0;i<n;i++) S[i]=false;
        //memset(T, false, sizeof(T)); //init set T
        for(int i=0;i<n;i++) T[i]=false;
        //memset(prev, -1, sizeof(prev)); //init set prev - for the alternating tree
        for(int i=0;i<n;i++) prev[i]=-1;
        for (x = 0; x < n; x++) //finding root of the tree
        {
            if (xy[x] == -1)
            {
                q[wr++] = root = x;
                prev[x] = -2;
                S[x] = true;
                break;
            }
        }
        for (y = 0; y < n; y++) //initializing slack array
        {
            slack[y] = lx[root] + ly[y] - cost[root][y];
            slackx[y] = root;
        }
        while (true) //main cycle
        {
            while (rd < wr) //building tree with bfs cycle
            {
                x = q[rd++]; //current vertex from X part
                for (y = 0; y < n; y++) //iterate through all edges in equality graph
                {
                    if (cost[x][y] == lx[x] + ly[y] && !T[y])
                    {
                        if (yx[y] == -1) break; //an exposed vertex in Y found, so
//augmenting path exists!
                        T[y] = true; //else just add y to T,
                        q[wr++] = yx[y]; //add vertex yx[y], which is matched
//with y, to the queue
                        add_to_tree(yx[y], x); //add edges (x,y) and (y,yx[y]) to the tree
                    }
                }

                if (y < n) break; //augmenting path found!
            }
            if (y < n) break; //augmenting path found!
            update_labels(); //augmenting path not found, so improve labeling
            wr = rd = 0;

            for (y = 0; y < n; y++)
            {
                //in this cycle we add edges that were added to the equality graph as a
//result of improving the labeling, we add edge (slackx[y], y) to the tree if
//and only if !T[y] && slack[y] == 0, also with this edge we add another one
//(y, yx[y]) or augment the matching, if y was exposed
                if (!T[y] && slack[y] == 0)
                {
                    if (yx[y] == -1) //exposed vertex in Y found - augmenting path exists!
                    {
                        x = slackx[y];
                        break;
                    }
                    else
                    {
                        T[y] = true; //else just add y to T,
                        if (!S[yx[y]])
                        {
                            q[wr++] = yx[y]; //add vertex yx[y], which is matched with
//y, to the queue
                            add_to_tree(yx[y], slackx[y]); //and add edges (x,y) and (y,
//yx[y]) to the tree
                        }
                    }
                }
            }
            if (y < n) break; //augmenting path found!
        }
        if (y < n) //we found augmenting path!
        {
            max_match++; //increment matching
//in this cycle we inverse edges along augmenting path
            for (int cx = x, cy = y, ty; cx != -2; cx = prev[cx], cy = ty)
            {
                ty = xy[cx];
                yx[cy] = cx;
                xy[cx] = cy;
            }
            augment(); //recall function, go to step 1 of the algorithm
        }
    }//end of augment() function
public:
    HungarianAlgorithm(int vv,int inf=1000000000)
    {
        N=vv;
        n=N;
        max_match=0;
        this->inf=inf;
        lx=new int[N];
        ly=new int[N];//labels of X and Y parts
        xy=new int[N];//xy[x] - vertex that is matched with x,
        yx=new int[N];//yx[y] - vertex that is matched with y
        slack=new int[N];//as in the algorithm description
        slackx=new int[N];//slackx[y] such a vertex, that l(slackx[y]) + l(y) - w(slackx[y],y) = slack[y]
        prev=new int[N];//array for memorizing alternating paths
        S=new bool[N];
        T=new bool[N];//sets S and T in algorithm
        cost=new int*[N];//cost matrix
        for(int i=0; i<N; i++)
        {
            cost[i]=new int[N];
        }
    }
    ~HungarianAlgorithm()
    {
        delete []lx;
        delete []ly;
        delete []xy;
        delete []yx;
        delete []slack;
        delete []slackx;
        delete []prev;
        delete []S;
        delete []T;
        int i;
        for(i=0; i<N; i++)
        {
            delete [](cost[i]);
        }
        delete []cost;
    }
    void setCost(int i,int j,int c)
    {
        cost[i][j]=c;
    }
    int* matching(bool first=true)
    {
        int *ans;
        ans=new int[N];
        for(int i=0;i<N;i++)
        {
            if(first) ans[i]=xy[i];
            else ans[i]=yx[i];
        }
        return ans;
    }
    int hungarian()
    {
        int ret = 0; //weight of the optimal matching
        max_match = 0; //number of vertices in current matching
        for(int x=0;x<n;x++) xy[x]=-1;
        for(int y=0;y<n;y++) yx[y]=-1;
        init_labels(); //step 0
        augment(); //steps 1-3
        for (int x = 0; x < n; x++) //forming answer there
            ret += cost[x][xy[x]];
        return ret;
    }
};
int main()
{
    int t,T=1;
    scanf("%d",&T);
    for(t=0;t<T;t++)
    {
        int n,i,j;
        scanf("%d",&n);
        HungarianAlgorithm h(n);
        int own[n],opposite[n];
        for(i=0;i<n;i++)
        {
            scanf("%d",&own[i]);
        }
        for(j=0;j<n;j++)
        {
            scanf("%d",&opposite[j]);
        }
        for(i=0;i<n;i++)
        {
            for(j=0;j<n;j++)
            {
                int v;
                if(own[i]==opposite[j]) v=1;
                else if(own[i]>opposite[j]) v=2;
                else v=0;
                h.setCost(i,j,v);
            }
        }
        int ans=h.hungarian();
        printf("Case %d: %d\n",t+1,ans);
    }
    return 0;
}

\end{minted}


\subsection{Maximum Bipartite Matching}
